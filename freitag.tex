\renewcommand{\konferenztag}{\freitag}
\newtimeslot{8:45}

%\abstractZwei{Tobias Preuß}{Projekt Umweltzone}{zwischen Open Data und OpenStreetMap}{%
%	In diesem Vortrag wird das Projekt "Umweltzone" vorgestellt. Dabei handelt es sich 
%	um eine kostenlose Open-Source-Anwendung für Android-Geräte, mit der sich Benutzer 
%	über die Lage der Umweltzonen verschiedener deutscher Städte informieren können. Die 
%	Herausforderung bestand in der Recherche, Beschaffung, Digitalisierung bzw. 
%	Umwandlung und Integration der Geoinformation zu den Umweltzonen. }
\abstractZwei{Tobias Preuß}{Projekt Umweltzone}{zwischen Open Data und OpenStreetMap}{%
	In diesem Vortrag wird das Projekt "Umweltzone" vorgestellt. Dabei handelt es sich 
	um eine kostenlose Open-Source-Anwendung für Android-Geräte, mit der sich Benutzer 
	über die Lage der Umweltzonen verschiedener deutscher Städte informieren können. 
	Die Herausforderung bestand in der Recherche, Beschaffung, Digitalisierung bzw. 
	Umwandlung und Integration der Geoinformation zu den Umweltzonen. 
	
	Im Vortrag wird auch über den Verlauf und Erfolg der Wochenaufgabe \emph{Umweltzonen} im August 2014 in der deutschen OSM-Community berichtet. }

\abstractZehn{Sven Böhme}{Öffentliche Projekte und Open Source}{Open Source Software in der öffentlichen Verwaltung am Beispiel der GDI-DE Testsuite}{%
	Anhand der GDI-DE Testsuite wird dargestellt, wie die GDI-DE die Strukturen von Open Source Projekten, 
	wie sie u.a. durch die OSGeo beschrieben werden, in Ihren Komponenten verwendet. Betriebsstelle GDI-DE 
	im Bundesamt für Kartographie und Geodäsie (BKG) wurde als Schnittstelle zwischen der fachlichen und der 
	technischen Umsetzung geschaffen. Der Vortrag bietet einen Einblick in die tägliche Arbeit der Betriebsstelle. 
	Die Ziele sowie die Hürden im täglichen Betrieb werden vorgestellt.}

%TODO Verlängerung überlegenswert
\abstractEins{Thomas Jakubicka}{Indoor Routing in Gebäuden des öffentlichen Verkehrs auf Basis von OpenStreetMap Daten}{}{%
	In unserem Vortrag möchten wir unser Konzept der Indoor-Navigation vorstellen und Erfahrungen und "Best Practices" 
	präsentieren die wir bei der Indoor Erfassung von Gebäuden des öffentlichen Verkehrs gemacht haben. }
%
% 9:15
\newtimeslot{9:15}
%
% zusätzliches vspace wegen mehrzeiliger Titel nötig
\abstractZwei{Marc Beiling}{Verkehrsmittelbezogene Erreichbarkeitsvisualisierung der Ruhr-Universität Bochum\vspace{0.3em}}{%
	Eine beispielhafte Mappinganwendung im Mobilitätskontext}{%
	Der Vortrag behandelt die Anwendung Verkehrsmittelbezogene Erreichbarkeitsvisualisierung für 
	Studierende der Ruhr-Universität Bochum, die als beispielhafte Umsetzung zu verstehen ist. 
	Die Herausforderungen beim Routing mit OSM-Daten werden aufgezeigt und Neulingen im 
	OSM/FOSS-Bereich wird ein Einblick in netzwerkbasierte Berechnungen/Visualiserungen 
	und dafür zu verwendende Komponenten ermöglicht.}

%\abstractZehn{Peter Löwe}{Das audiovisuelle Erbe der OSGeo-Projekte }{Referenzfall GRASS GIS - und Star Trek}{%
%	Dieser Vortrag diskutiert die Problematik der Archivierung aller Aspekte von Open Source-Projekten und stellt 
%	das AV-Portal der Technischen Informationsbibliothek (TIB) vor. Das Portal ist eine zukunftssichere Alternative 
%	zum häufig praktizierten eher flüchtigen Datenaustausch über Plattformen wie Youtube oder Slideshare 
%	und kann nicht nur Metadaten eines Videos indexieren, sondern auch die gesprochene Sprache, Texteinblendungen oder Bildinformationen.}

\abstractZehn{Peter Löwe}{Das audiovisuelle Erbe der OSGeo-Projekte }{Referenzfall GRASS GIS - und Star Trek}{%
	Dieser Vortrag diskutiert die Problematik der Archivierung aller Aspekte von Open Source-Projekten und 
	stellt das AV-Portal der Technischen Informationsbibliothek (TIB) vor. Das Portal ist eine zukunftssichere 
	Alternative zum häufig praktizierten eher flüchtigen Datenaustausch über Plattformen wie Youtube oder 
	Slideshare und kann nicht nur Metadaten eines Videos indexieren, sondern auch die gesprochene Sprache, 
	Texteinblendungen oder Bildinformationen.
	Durch die Verbindung eines DOI mit einem Media Fragment Identifier wird die sekundengenaue Zitierfähigkeit der Materialien gewährleistet. 
	
	Der Nutzen wird am Beispiel der erfolgreichen digitalen Erschließung des GRASS GIS Videos des U.S. 
	Army Corps of Engineers Research Laboratory (CERL) aus dem Jahr 1987 demonstriert.}

%\abstractEins{Arndt Brenschede}{Neues zu BRouter}{Mehr als nur Outdoor-Navigation}{%
%	Dieser Vortrag stellt einige der Neuerungen aus dem letzten Jahr beim Projekt BRouter vor, 
%	mit einem besonderen Augenmerk auf den neuen Möglichkeiten im Hinblick auf Spezialanwendungen durch das neue, flexiblere Datenmodell.}
\abstractEins{Arndt Brenschede}{Neues zu BRouter}{Mehr als nur Outdoor-Navigation}{%
	Dieser Vortrag stellt einige der Neuerungen aus dem letzten Jahr beim Projekt BRouter vor, 
	die selbst die regelmässigen Nutzer noch kaum kennen.
	Beispiele sind steigungsabhängige Weg-Präferenzen, schnellere Neuberechnungen und die Integration der 30m-SRTM Höhendaten.
	
	Die wichtigste Neuerung aber ist ein neues, flexibleres internes Datenmodell. 
	Dadurch werden sowohl Spezialanwendungen im Wegenetz möglich, etwa im Hinblick auf Barrierefreiheit, 
	Einsatzfahrzeuge oder Schwerlastverkehr, aber auch Anwendungen in ganz anderen Netzen wie dem Schienen, Fluss- oder Stromnetz.}
%
% 9:45
\newtimeslot{9:45}
%
%\abstractZwei{Friedrich Müller}{Linked Data basierter Explorer}{%
%	Ein Assistenzsystem zur Erforschung von Umweltdaten für krebsrelevante Ursache-Wirkungs-Beziehungen im raum-zeitlichen Kontext}{%
%	Die Webapplikation, basierend auf Linked Data, ermöglicht als Assistenzsystem die Erforschung von 
%	Ursache-Wirkungs-Beziehungen zwischen Krebstypen und Umweltstoffen innerhalb einer vordefinierten 
%	Region durch dynamische Geovisualisierungen.}
\abstractZwei{Friedrich Müller}{Linked Data basierter Explorer}{%
	Ein Assistenzsystem zur Erforschung von Umweltdaten für krebsrelevante Ursache-Wirkungs-Beziehungen im raum-zeitlichen Kontext}{%
	Aktuell ist eine Vielzahl krebsbezogener Informationen als Open-Data verfügbar. Allerdings sind diese Datensätze oft 
	nicht aggregiert, in unterschiedlichen Formate oder teilweise nur unter erhöhtem Arbeitsaufwand zugänglich. Die hier 
	präsentierte Webanwendung vereinfacht, auf Basis von Linked Data und weiteren semantischen Technologien, die Erreichbarkeit 
	krebsrelevanter Informationen. Neben der Möglichkeit sich über die krebsbezogenen Ursache-Wirkungs-Beziehungen zu informieren, 
	ist der Hauptnutzen der Applikation die ermöglichte Erforschung von Umweltdatenim Zusammenhang 
	mit epidemiologischen Datensätzen u.\,a. per Geovisualisierungen für die Beispielregion Westfalen-Lippe.
	
	Der Workflow basiert komplett auf freier Software/Daten, (z.\,B. Protegé, Apache Jena, R, OSM und Leaflet). 
	Produkte des Projektes sind neben dem Quellcode zur Wiederverwendung auf Github zugänglich.}


% kann man noch länger machen, wenn man keine Werbung hat
\abstractZehn{Oliver Tonnhofer und Till Adams}{Der schwere Werdegang zu einem FOSSGIS-Open Source Projekt}{}{%
	OpenSource machen ist einfach. Ein bisschen Code geschrieben, einen schicken Lizenz-Header oben drüber gepastet 
	und ab damit auf Git oder eine andere hippe Plattform. Aber damit ist es dann meistens doch nicht getan. 
	Der Vortrag beschreibt warum.}

\abstractEins{Christoph Hormann (Moderation)}{OSM Lightning Talks II}{}{}
%
% 11:00
\newtimeslot{11:00}
%
%\abstractZwei{Daniel Kastl}{Location-based Task Management}{%
%	Standortbezogene Aufgabenverwaltung für mobile Arbeitsplätze}{%
%	FOSS4G Software kann bei vielerlei Aufgaben mit Raumbezug helfen, Arbeitsabläufe zu verbessern und zu 
%	optimieren, und die Mitarbeiter bei Ihrer täglichen Arbeit zu entlasten. Dieser Vortrag stellt ein Konzept vor für "Location-based Task Management".}

\abstractZwei{Daniel Kastl}{Location-based Task Management}{Standortbezogene Aufgabenverwaltung für mobile Arbeitsplätze}{%
	FOSS4G Software kann bei vielerlei Aufgaben mit Raumbezug helfen, Arbeitsabläufe zu verbessern und zu optimieren, und 
	die Mitarbeiter bei Ihrer täglichen Arbeit zu entlasten. Dieser Vortrag stellt ein Konzept vor für "Location-based Task Management".
	
	Mit der Optimierung von Arbeitsabläufen und Tourenplanung im Hinterkopf, und auf Grundlage verfügbarer Open-Source-Software 
	und offenen Standards haben wir begonnen, eine Task-Management-Software zu entwickeln. In dieser Präsentation 
	werden Sie Details zu unserem Konzept zu erfahren, und wie dies dazu beitragen kann, Dienstleistungen vor Ort 
	zu erleichtern, die Servicequalität zu verbessern und die Effizienz zu steigern.}

\abstractZehn{Felix Kunde}{3D GIS Stack aus OpenSource Komponenten}{}{%
	In den letzten Jahren hat die dritte Dimension auch Einzug in den gängigen FOSSGIS Lösungen (PostGIS, QGIS, OpenLayers etc.) 
	gehalten, so dass mittlerweile ein kompletter 3D-GIS-Stack aus OpenSource Lösungen realisiert werden kann. 
	Das wichtigste Ziel der hier vorgestellten Projekte ist die Interaktion mit 3D-Webkarten. Der Anwender soll 
	in der Lage sein, mit den 3D-Modellen über das Web arbeiten zu können und sie nicht nur zu betrachten.}

% Variante 1 ist auch gut
%\abstractEins{Roland Olbricht}{Schatzsuche in OpenStreetMap}{}{%
%	Mit der Overpass API lassen sich auch ungewöhnliche Daten in OpenStreetMap finden -- und bewundern, 
%	ihnen nachspüren oder sie korrigieren. Neben einer Präsentation der Ergebnisse des Vergleiches von 
%	Bonn und Münster werden dabei auch ausführlich die verwendeten Overpass-API-Abfragen vorgestellt, 
%	sodass jeder die Abfragen leicht für seine Stadt wiederholen oder inhaltlich auf seine Bedürfnisse anpassen kann.}
\abstractEins{Roland Olbricht}{Schatzsuche in OpenStreetMap}{}{%
	Mit der Overpass API lassen sich auch ungewöhnliche Daten in OpenStreetMap finden -- und bewundern, 
	ihnen nachspüren oder sie korrigieren. 
	
	Wir untersuchen zwei Großstädte, Bonn und Münster, mit verschiedenen Ansätzen auf ungewöhnliche Daten. 
	Was bleibt, wenn man alle Objekte weglässt, die sich anhand ihrer Tags einer klaren Kategorie zuordnen lassen? 
	Wo fehlen wahrscheinlich noch Briefkästen? Welche Brücken haben keine Höhenangabe? 
	Wo steht in den Tags eines Elements vermutlich Spam? Welche Restaurants (und sonstige Points of Interest) sind verdächtig alt?
	
	Neben einer Präsentation der Ergebnisse werden dabei auch ausführlich die verwendeten Overpass-API-Abfragen vorgestellt, 
	sodass jeder die Abfragen leicht für seine Stadt wiederholen oder inhaltlich auf seine Bedürfnisse anpassen kann.}
%
% 11:30
\newtimeslot{11:30}
%
%\abstractZwei{Tobias Sauerwein}{MapFish Print V3: Printing maps like a boss}{The next generation of printed maps}{%
%	Dieser Vortrag stellt die neue Version von MapFish Print vor und spricht Themen an wie die neuen Features und 
%	deren Nutzung, die Erstellung von Templates mit dem Report-Designer von JasperReports, Upgrade von der 
%	vorherigen Version, Skalierbarkeit und Erweiterung durch eigene Module.}
\abstractZwei{Tobias Sauerwein}{MapFish Print V3 -- Printing maps like a boss\vspace{0.3em}}{The next generation of printed maps}{%
	Das Projekt MapFish Print besteht aus einer Bibliothek und einer Web-Anwendung zum Druck von Reports mit Karten, 
	wobei eine Vielfalt von Quellen unterstützt wird, z.\,B. WMS, WMTS, OpenStreetMap-Kacheln, WFS oder GeoJSON.
	
	Dieser Vortrag stellt die neue Version von MapFish Print vor und spricht Themen an wie die neuen Features und 
	deren Nutzung, die Erstellung von Templates mit dem Report-Designer von JasperReports, Upgrade von der 
	vorherigen Version, Skalierbarkeit und Erweiterung durch eigene Module.}

%\abstractZehn{Elisabeth Leu}{Cesium -- der 3D-Globus im Web}{Open Source Technologie in drei Dimensionen}{%
%	Cesium ist ein performantes und interoperables Tool für die Visualisierung von Daten im dreidimensionalen Kontext. 
%	Stichworte zum Vortrag: 3D - JavaScript - Open Source - WebGL - Zeitabhängige Darstellung - 
%	OGC Standards - Openlayers 3 API - Demos und Beispiele.}
\abstractZehn{Elisabeth Leu}{Cesium -- der 3D-Globus im Web}{Open Source Technologie in drei Dimensionen}{%
	Cesium ist eine JavaScript-Bibliothek, mit der 3D-Globen für das Web erstellt werden können. 
	Cesium nutzt WebGL und unterstützt ausserdem OGC-Standards wie WMS oder WMTS.
	
	Im Vortrag werden folgende Fragen beantwortet: Was kann Cesium? Performante mehrdimensionale Visualisierung im Web -- was steckt dahinter? 
	Wo wird überall Cesium eingesetzt? Ausserdem wird die Kombination von Openlayers3 mit Cesium vorgestellt 
	und ein Ausblick über die künftige Entwicklung gegeben.}

%\abstractEins{Robert Buchholz}{FlatMatch -- Online-Wohnungssuche mit OSM-Daten}{Können wir tausende deutsche Vermieter zu OSM-Mitwirkenden machen?}{%
%	Interaktive 3D-Wohnungsbesichtigung im Browser auf Basis von OSM-Daten ermöglicht ein 
%	besseres Bewerten freier Wohnungen und deren Umfeld. Und sie macht Vermieter zu OSM-Stakeholdern. 
%	Die Mitarbeit kommerzieller Nutzer an OSM hängt jedoch allein davon ab, ob sie dadurch einen 
%	wirtschaftlichen Mehrwert erhalten. Diese Vortrag stellt ein Projekt vor, dass potentiell 
%	zehntausende kleine und mittlere Unternehmen in Deutschland diesen Mehrwert bietet.}
\abstractEins{Robert Buchholz}{FlatMatch -- Online-Wohnungssuche mit OSM-Daten}{Können wir tausende deutsche Vermieter zu OSM-Mitwirkenden machen?}{%
	FlatMatch ist eine Web-Anwendung, um Mietwohnungen direkt im Webbrowser virtuell zu besichtigen und so eine geeignete neue Wohnung zu finden. 
	Hierzu wird nicht nur die Wohnung interaktiv dreidimensional dargestellt. Auch die Aussicht aus der Wohnung 
	wird -- auf Basis von OSM-3D-Gebäuden und TileMaps -- dargestellt, um einen besseren Einblick in das Wohnumfeld zu geben. 
	
	Dies vereinfacht für Mieter die Wohnungssuche und erlaubt Vermietern, den Personalaufwand für Besichtigungen zu reduzieren, 
	und ihre Wohnungen auch auf Basis deren Umfelds zu vermarkten.
	Die Mitarbeit kommerzieller Nutzer an OSM hängt jedoch allein davon ab, ob sie dadurch einen 
	wirtschaftlichen Mehrwert erhalten. FlatMatch könnte zehntausende kleine und mittlere Unternehmen in Deutschland 
	diesen Mehrwert (und eine Motivation, ihre Umgebung zu mappen) bieten.}
%
% 12:30
\newtimeslot{12:30}
%
\abstractAula{Marco Lechner}{Abschlußveranstaltung}{}{Die Abschlussveranstaltung der FOSSGIS-Konferenz 2015 blickt auf die gelaufene Konferenz und auf die nächste. Wir dürfen alle gespannt sein. In diesem Rahmen findet auch die Verlosung statt, an der alle teilnehmen, die den Teilnehmerfeedbackbogen ausgefüllt haben.}
%
% 13:30
\newsmalltimeslot{13:30}
%
\abstractAula{Katja Haferkorn}{Sektempfang}{}{Sektempfang am FOSSGIS-Stand zum Ausklang der Konferenz. Der FOSSGIS e.V. - Veranstalter der Konferenz - gibt einen Sektempfang. Mitglieder des Vereins, Freunde und Interessierte sind herzlich eingeladen.}
