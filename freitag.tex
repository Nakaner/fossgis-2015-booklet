\renewcommand{\konferenztag}{\freitag}
\label{freitag}
\newtimeslot{8:45}

\abstractZwei{Tobias Preuß}{Projekt Umweltzone}{zwischen Open Data und OpenStreetMap}{In diesem Vortrag wird das Projekt "Umweltzone" vorgestellt. Dabei handelt es sich um eine kostenlose Open-Source-Anwendung für Android-Geräte, mit der sich Benutzer über die Lage der Umweltzonen verschiedener deutscher Städte informieren können. Die Herausforderung bestand in der Recherche, Beschaffung, Digitalisierung bzw. Umwandlung und Integration der Geoinformation zu den Umweltzonen. }
\abstractZehn{Sven Böhme}{Öffentliche Projekte und Open Source}{Open Source Software in der öffentlichen Verwaltung am Beispiel der GDI-DE Testsuite}{Anhand der GDI-DE Testsuite wird dargestellt, wie die GDI-DE die Strukturen von Open Source Projekten, wie sie u.a. durch die OSGeo beschrieben werden, in Ihren Komponenten verwendet. Betriebsstelle GDI-DE im Bundesamt für Kartographie und Geodäsie (BKG) wurde als Schnittstelle zwischen der fachlichen und der technischen Umsetzung geschaffen. Der Vortrag bietet einen Einblick in die tägliche Arbeit der Betriebsstelle. Die Ziele sowie die Hürden im täglichen Betrieb werden vorgestellt.}
\abstractEins{Thomas Jakubicka}{Indoor Routing in Gebäuden des öffentlichen Verkehrs auf Basis von OpenStreetMap Daten}{}{In unserem Vortrag möchten wir unser Konzept der Indoor Navigation vorstellen und Erfahrungen und „Best Practices“ präsentieren die wir bei der Indoor Erfassung von Gebäuden des öffentlichen Verkehrs gemacht haben. }
%
% 9:15
\newtimeslot{9:15}
%
\abstractZwei{Marc Beiling}{Verkehrsmittelbezogene Erreichbarkeitsvisualisierung der Ruhr-Universität Bochum }{Eine beispielhafte Mappinganwendung im Mobilitätskontext}{Der Vortrag behandelt die Anwendung Verkehrsmittelbezogene Erreichbarkeitsvisualisierung für Studierende der Ruhr-Universität Bochum, die als beispielhafte Umsetzung zu verstehen ist. Die Herausforderungen beim Routing mit OSM-Daten werden aufgezeigt und Neulingen im OSM/FOSS-Bereich wird ein Einblick in netzwerkbasierte Berechnungen/Visualiserungen und dafür zu verwendende Komponenten ermöglicht.}
\abstractZehn{Peter Löwe}{Das audiovisuelle Erbe der OSGeo-Projekte }{Referenzfall GRASS GIS - und Star Trek}{Dieser Vortrag diskutiert die Problematik der Archivierung aller Aspekte von Open Source-Projekten und stellt das AV-Portal der Technischen Informationsbibliothek (TIB) vor. Das Portal ist eine zukunftssichere Alternative zum häufig praktizierten eher flüchtigen Datenaustausch über Plattformen wie Youtube oder Slideshare und kann nicht nur Metadaten eines Videos indexieren, sondern auch die gesprochene Sprache, Texteinblendungen oder Bildinformationen.}
\abstractEins{Arndt Brenschede}{Neues zu BRouter}{- mehr als nur Outdoor-Navigation -}{Dieser Vortrag stellt einige der Neuerungen aus dem letzten Jahr beim Projekt BRouter vor, mit einem besonderen Augenmerk auf den neuen Möglichkeiten im Hinblick auf Spezialanwendungen durch das neue, flexiblere Datenmodell.}
%
% 9:45
\newtimeslot{9:45}
%
\abstractZwei{Friedrich Müller}{Linked Data basierter Explorer}{Ein Assistenzsystem zur Erforschung von Umweltdaten für krebsrelevante Ursache-Wirkungs-Beziehungen im raum-zeitlichen Kontext}{Die Webapplikation, basierend auf Linked Data, ermöglicht als Assistenzsystem die Erforschung von Ursache-Wirkungs-Beziehungen zwischen Krebstypen und Umweltstoffen innerhalb einer vordefinierten Region durch dynamische Geovisualisierungen.}
\abstractZehn{Oliver Tonnhofer}{Der schwere Werdegang zu einem FOSSGIS-Open Source Projekt}{}{OpenSource machen ist einfach. Ein bisschen Code geschrieben, einen schicken Lizenz-Header oben drüber gepastet und ab damit auf Git oder eine andere hippe Plattform. Aber damit ist es dann meistens doch nicht getan. Der Vortrag beschreibt warum.}
\abstractEins{Christoph Hormann}{OSM Lightning Talks II}{}{xxx???}
%
% 11:00
\newtimeslot{11:00}
%
\abstractZwei{Daniel Kastl}{Location-based Task Management}{Standortbezogene Aufgabenverwaltung für mobile Arbeitsplätze}{FOSS4G Software kann bei vielerlei Aufgaben mit Raumbezug helfen, Arbeitsabläufe zu verbessern und zu optimieren, und die Mitarbeiter bei Ihrer täglichen Arbeit zu entlasten. Dieser Vortrag stellt ein Konzept vor für "Location-based Task Management".}
\abstractZehn{Felix Kunde}{3D GIS Stack aus OpenSource Komponenten}{}{In den letzten Jahren hat die dritte Dimension auch Einzug in den gängigen FOSSGIS Lösungen (PostGIS, QGIS, OpenLayers etc.) gehalten, so dass mittlerweile ein kompletter 3D-GIS-Stack aus OpenSource Lösungen realisiert werden kann. Das wichtigste Ziel der hier vorgestellten Projekte ist die Interaktion mit 3D-Webkarten. Der Anwender soll in der Lage sein, mit den 3D-Modellen über das Web arbeiten zu können und sie nicht nur zu betrachten.}
\abstractEins{Roland Olbricht}{Schatzsuche in OpenStreetMap}{}{Mit der Overpass API lassen sich auch ungewöhnliche Daten in OpenStreetMap finden - und bewundern, ihnen nachspüren oder sie korrigieren. Neben einer Präsentation der Ergebnisse des Vergleiches von Bonn und Münster werden dabei auch ausführlich die verwendeten Overpass-API-Abfragen vorgestellt, so dass jeder die Abfragen leicht für seine Stadt wiederholen oder inhaltlich auf seine Bedürfnisse anpassen kann.}
%
% 11:30
\newtimeslot{11:30}
%
\abstractZwei{Tobias Sauerwein}{MapFish Print V3: Printing maps like a boss}{The next generation of printed maps}{Dieser Vortrag stellt die neue Version von MapFish Print vor und spricht Themen an wie die neuen Features und deren Nutzung, die Erstellung von Templates mit dem Report-Designer von JasperReports, Upgrade von der vorherigen Version, Skalierbarkeit und Erweiterung durch eigene Module.}
\abstractZehn{Elisabeth Leu}{Cesium - der 3D-Globus im Web}{Open Source Technologie in drei Dimensionen}{Cesium ist ein performantes und interoperables Tool für die Visualisierung von Daten im dreidimensionalen Kontext. Stichworte zum Vortrag: 3D - JavaScript - Open Source - WebGL - Zeitabhängige Darstellung - OGC Standards - Openlayers 3 API - Demos und Beispiele.}
\abstractEins{Robert Buchholz}{FlatMatch: Online-Wohnungssuche mit OSM-Daten}{Können wir tausende deutsche Vermieter zu OSM-Mitwirkenden machen?}{Interaktive 3D-Wohnungsbesichtigung im Browser auf Basis von OSM-Daten ermöglicht ein besseres Bewerten freier Wohnungen und deren Umfeld. Und sie macht Vermieter zu OSM-Stakeholdern. Die Mitarbeit kommerzieller Nutzer an OSM hängt jedoch allein davon ab, ob sie dadurch einen wirtschaftlichen Mehrwert erhalten. Diese Vortrag stellt ein Projekt vor, dass potentiell zehntausende kleine und mittlere Unternehmen in Deutschland diesen Mehrwert bietet.}
%
% 12:30
\newtimeslot{12:30}
%
\abstractAula{Marco Lechner}{Abschlußveranstaltung}{}{Die Abschlussveranstaltung der FOSSGIS-Konferenz 2015 blickt auf die gelaufene Konferenz und auf die nächste. Wir dürfen alle gespannt sein. In diesem Rahmen findet auch die Verlosung statt, an der alle teilnehmen, die den Teilnehmerfeedbackbogen ausgefüllt haben.}
%
% 13:30
\newtimeslot{13:30}
%
\abstractAula{Katja Haferkorn}{Sektempfang}{}{Sektempfang am FOSSGIS-Stand zum Ausklang der Konferenz. Der FOSSGIS e.V. - Veranstalter der Konferenz - gibt einen Sektempfang. Mitglieder des Vereins, Freunde und Interessierte sind herzlich eingeladen.}
