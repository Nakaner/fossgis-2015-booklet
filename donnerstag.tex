\renewcommand{\konferenztag}{\donnerstag}
\newtimeslot{9:00}


% 9:00
%
%\abstractZwei{Otto Dassau und Sören Gebbert}{GRASS-Funktionalität in QGIS nutzen}{Möglichkeiten der optimalen Interaktion von QGIS und GRASS}{Seit 2005 ist GRASS GIS in QGIS über das GRASS Plugin integriert und stellt hunderte Analysemethoden über QGIS bereit - ergänzend mittlerweile auch über das Processing und WPS Plugin. Wir vergleichen die drei Varianten und schauen in die Zukunft. }
\abstractZwei{Otto Dassau und Sören Gebbert}{GRASS-Funktionalität in QGIS nutzen}{}{%
	Seit 2005 ist GRASS GIS in QGIS über das GRASS Plugin integriert und stellt hunderte Analysemethoden über QGIS bereit -- ergänzend mittlerweile auch über das Processing und WPS Plugin. 
	
	Wir starten in diesem Vortrag einen Vergleich der drei Möglichkeiten zur Interaktion von GRASS und QGIS. Anhand von Beispielen stellen wir die Vor- und Nachteile dar und versuchen aufzuzeigen, welche Variante in welcher Situation die beste Lösung darstellt. Wann ist es sinnvoll das Processing Plugin zu verwenden, wann sollte man auf das bewährte GRASS Plugin setzen und welche Stärken und Schwachstellen bietet die Integration von GRASS Funktionalität über das WPS Plugin.
	
	Abschließend stellen wir die Aktuelle Entwicklung der Interaktion zwischen GRASS und QGIS vor und schauen in die Zukunft.}


\abstractZehn{Andreas Hocevar}{OpenLayers 3}{Stand, Neues und Zukünfiges}{%
	OpenLayers 3 wird vorgestellt, zahlreiche Beispiele zeigen die Verwendung. Unterschiede zur Vorgängerversion 
	werden aufgezeigt und auch wo und warum OpenLayers 3 anders ist. Aktuelle Entwicklungen, wie etwa das OL3-Cesium 
	Project, welches die dritte Dimension für OpenLayers zuänglich macht, werden präsentiert und ein Blick in die 
	zukünftige Entwicklung gewagt. }

%\abstractEins{Falk Zscheile}{OSM und die Kunst die Welt zu ordnen}{%
%	Sprachliche und logische Aspekte des Taggingschemas in OpenStreetMap}{Auseinandersetzung mit der Frage des "richtigen" Taggings, 
%	also der "richtigen" Wiedergabe der Wirklichkeit mittels Tags, aus Sicht der Logik und sprachwissenschaftlicher Erkenntnisse.
%	}
	
\abstractEins{Falk Zscheile}{OSM und die Kunst die Welt zu ordnen}{Sprachliche und logische Aspekte des Taggingschemas in OpenStreetMap}{%
	Anders als bei der klassischen Geodatenerfassung gibt es bei OSM keine hierarchisches gegliedertes System zur Kennzeichnung der erfassten Daten. 
	Über die Frage des "richtigen" Taggings, wird sehr häufig gestritten. Oft verlaufen die Diskussionen ohne Ergebnis im Sande.
	
	Dabei zeigt sich, dass häufig schon bei der Diskussion eines Problems aneinander vorbeigeredet wird, da der Kontext 
	des Diskussionspartners nicht berücksichtigt wird. Dies zeigt sich häufig auch in einer mangelnden Auseinandersetzung mit 
	Grundprinzipien der Logik. Zudem bleibt unberücksichtigt, dass sich die Tagfindung aufgrund der "on the ground rule" an 
	der Alltagsbedeutung von Begriffen orientiert, in den anschließenden Diskussionen aber eine feste Bedeutung im Sinne 
	wissenschaftlicher Definitionen unterstellt wird.
	}
%

\newtimeslot{9:30}
% 9:30
%
\abstractZwei{Sören Gebbert}{Zeitreihenanalyse mit GRASS GIS}{GRASS das temporale Open Source GIS}{%
	Durch die Integration der Zeit als neue Dimension in GRASS GIS stehen nun über 45 Werkzeuge zur Zeitreihenalayse bereit. 
	Eine Auswahl an Werkzeugen für die Analyse, Verarbeitung und Visualisierung von Raster- und Vektor-Zeitreihen wird vorgestellt.}

%\abstractZehn{Dominik Helle}{Ein Duett: OpenLayers 3 im Zusammenspiel mit AngularJS}{}{AngularJS im Zusammenspiel mit OpenLayers3. Zwei mächtige Javascript-Bibliotheken treffen aufeinander. Potentiale des Zusammenspiels beider Komponenten werden anhand von Praxisbeispielen aufgezeigt. }
\abstractZehn{Dominik Helle}{Ein Duett -- OpenLayers 3 im Zusammenspiel mit AngularJS}{}{%
	AngularJS ist ein mächtiges Open-Source-Framework mit der clientseitige Webanwendungen nach dem MVC-Prinzip erstellt werden können. 
	OpenLayers ist wohl eine der bekanntesten Javascript-Bibliotheken um Karten im Netz nutzen zu können.
	
	Im Vortrag werden Möglichkeiten und Anregungen des Zusammenspiels dieser beiden Komponenten aufgezeigt. 
	Ein erstes Aufeinandertreffen der beiden Komponenten gab es für den Autor dieses Vortrags im Rahmen eines 
	Projektes in dem eine Kartenanwendung umgesetzt wurde. Besonderes Augenmerk lag bei der Entwicklung auf der 
	schnellen und leichten anpassbar- und Wiederverwendbarkeit der Komponenten.}

% Alles weitere, was im Abstract steht ist relativ logisch. Außerdem ist der Rest des Absatzes auch mit inhaltlichen Fehlern gespickt, 
% die man niemanden lesen lassen möchte.
%\abstractEins{Nicole Beringer}{OSM auf Räder}{}{Sind die OSM Karten soweit, um den Wettbewerb mit kommerziellen Karten im Autonavigationsbereich aufzunehmen?}
\abstractEins{Nicole Beringer}{OSM auf Räder}{}{%
	Sind die OSM-Karten soweit, um den Wettbewerb mit kommerziellen Karten im Autonavigationsbereich aufzunehmen?	
	Die Software-Unternehmen aus dem automotiven Bereich verfolgen die OSM-Entwicklung mit großer 
	Aufmerksamkeit und stehen auf der Spitze der Implementierung dieser neuen Technologie.	
	}
%

\newtimeslot{10:00}
% 10:00
%
\abstractZwei{Stefan Ziegler}{Kreisbogen in QGIS}{}{Neben den gängigen Vektorelementen (Punkt, Linie, Polygon) unterstützt QGIS neu Kreisbogengeometrien. Es ist jetzt möglich solche Geometrietypen (ST\_CircularString, ST\_CompoundCurve, ST\_CurvePolygon etc.) aus einer Postgis-Datenbank zu lesen, anzuzeigen, zu editieren und wieder zu speichern.}

\abstractZehn{Christian Mayer und Marc Jansen}{GeoExt}{Ein ExtJS basiertes Rich-Client-WebGIS-Framework in Zeiten von AngularJS und Konsorten}{%
	Der Vortrag stellt die neueste Version von GeoExt vor und zeigt auf, welches Handwerkszeug dem Entwickler 
	hier bereitsgestellt wird. Unterschiede zwischen ExtJS und anderen Bibliotheken werden benannt, dies kann 
	als Diskussionsgrundlage für die Wahl einer Bibliothek dienen. Schwerpunkt ist die Betrachtung der zukünftigen 
	Entwicklung von GeoExt.}

%\abstractEins{Robert Klemm}{Automatisierte OSM Aufbereitung und Analyse von LKW-Mautstrecken in Deutschland}{}{Das beschriebene Tool und die damit erzeugten Mautdaten erleichtern dem OSM-Mappern die Arbeit. Für ihn validiert das Tool durch Routinganalyse die OSM-Daten, basierend auf dem Tagging-Schema mit mautbezogenen Routingparametern, wie Fahrzeug-, Achs-, Gewichtsklasse und Betreiber. Es lässt sich beispielsweise die schnellste und günstigste Strecke, bezogen auf die zu entrichtende Straßenmaut, ermitteln und grafisch anzeigen.}
\abstractEins{Robert Klemm}{Automatisierte OSM Aufbereitung und Analyse von LKW-Mautstrecken in Deutschland}{}{%
	Das beschriebene Tool erstellt mautpflichtigen Strecken anhand von Mautpunkten der Bundesanstalt für Straßenwesen (BASt). 
	Die Daten der BASt bezüglich der Strecken werden so umgewandelt, 
	dass sie in OSM abgelegt werden können. Dazu wird ein Taggingschema vorgeschlagen.
	
	Das beschriebene Tool und die damit erzeugten Mautdaten erleichtern dem Mappern die Arbeit. Für sie 
	validiert das Tool durch Routinganalyse die OSM-Daten, basierend auf dem Taggingschema mit mautbezogenen 
	Routingparametern, wie Fahrzeug-, Achs-, Gewichtsklasse und Betreiber. Es lässt sich beispielsweise die 
	schnellste und günstigste Strecke, bezogen auf die zu entrichtende Straßenmaut, ermitteln und grafisch 
	anzeigen.}
%

\newtimeslot{11:15}
% 11:15
%
\abstractZwei{Mila Frerichs}{Geospatial Ruby}{}{%
	Der Talk gibt einen Überblick darüber, was mit Ruby im Geo Bereich möglich ist. Viele große erfolgreiche Webprojekte sind mit Ruby und dem dazugehörigen Webframework Rails umgesetzt worden. 
	
	 Der Fokus dieses Vortrags liegt auf drei wichtigen Bibliotheken terraformer.rb, rgeo und SimpleTile.}

%\abstractZehn{Jörg Thomsen}{MapServer Pro-Tipps}{Das \emph{Do what I want} Know How für den UMN MapServer}{%
%	Über das Basiswissen hinaus gehende Funktionen des MapServers, wie PHP-Fassaden, JOINs mit Datei basierten Datenquellen, ColorRanges, Buildt-In Openlayers-Client, Cluster von Punktobjekten, Geomtransform, IP-Zugriffsbeschränkung sind Themen des Vortrages. }
\abstractZehn{Jörg Thomsen}{MapServer Pro-Tipps}{Das \emph{Do what I want} Know How für den UMN MapServer}{%
	MapServer ist bekannt für seine Zuverlässigkeit, Performance und Stabilität. Gerne wird aber auch über die Konfiguration mit den Mapfiles gestöhnt. Dabei bietet gerade diese Art der Konfiguration Möglichkeiten, die tief in der Dokumentation versteckt sind und die bei kniffligen Augaben wirklich hilfreich sein können. Die meisten Anwender kennen nur die Basis-Konfigurationen, aber wer weiß schon dass es einen eingebauten OpenLayers-Client gibt, der es ermögicht nur mit einem Mapfile eine interakive Karte zu veröffentlichen, wie man Linien mit Pfeilspitzen versieht, Geometrien glättet oder den Zugriff auf MapServer-WMS direkt im Mapfile auf bestimmte IP-Adressen beschränken kann?
	
	Weitere Themen des Vortrags sind PHP-Fassaden, JOINs mit Datei basierten Datenquellen, ColorRanges, Cluster von Punktobjekten, Geomtransform und IP-Zugriffsbeschränkungen.}

%\abstractEins{Claas Leiner}{Daten aus OSM extrahieren und in QGIS weiterverarbeiten }{Am Beispiel der Verteilung von Windanlagen in Deutschland}{Die OSM-Daten enthalten viele umweltrelevante Informationen, die auf den veröffentlichten Webkarten nicht offensichtlich erkennbar sind. Mit Hilfe von QGIS lassen sich diese Informationen auswerten und in neue aussagekräftigen Karten umsetzen.}
\abstractEins{Claas Leiner}{Daten aus OSM extrahieren und in QGIS weiterverarbeiten}{Am Beispiel der Verteilung von Windanlagen in Deutschland}{%
	Der Vortrag erläutert die Fragestellung am Beispiel der Verteilung von Windkraftanlagen in Deutschland. Angefangen vom Import der deutschlandweiten OSM-Daten in eine PostGis-Datenbank, über die Abfrage der WKA-Standorte bis hin zur Präsentation der Verteilung in einer farbigen Flächendichtekarte, die aus einer interpolierten Rasteroberfläche erzeugt worden ist, welche die Anzahl der Windanlagen im Umkreis von 20\,km darstellt. Dabei kamen Abfrage- und Geoverarbeitungswerkzeuge aus QGIS sowie das GRASS-Modul zur Spline-Interpolation (v.surf.rst) zur Anwendung.}
%

\newtimeslot{11:45}
% 11:45
%
%\abstractZwei{Alexey Valikov}{Jsonix: OGC Web Services in JSON}{Nie mehr XML in JavaScript-Apps verarbeiten, mit Jsonix spricht JSON mit den OGC Web Services }{Wie kann man mit den OGC Web Services in reinem JSON (statt XML) sprechen? Mit Jsonix, einem mächtigen JavaScript-Tool für XML<->JSON Konvertierung. Es gibt Live-Demos von WMS, WFS, CSW sowie OL3 WPS Client.}
\abstractZwei{Alexey Valikov}{Jsonix: OGC Web Services in JSON}{}{%
	JSON hat wahrscheinlich XML schon längst als "lingua franca" ersetzt. 
	Es ist viel leichtgewichtiger und einfacher zu verwenden als XML, vor 
	allem in den JavaScript-basierten Web-Apps.
	
	Das Web GIS Umfeld wird von JavaScript-Bibliotheken wie OpenLayers und Leaflet dominiert. 
	Für die gehört JSON sowieso zur Muttersprache. Aber die OGC-Standards sind fast alle XML-basiert 
	und durch XML Schemata spezifiziert.
	
	Jsonix ist eine Open-Source Bibliothek für die XML<->JS Konvertierung, die genau das möglich macht. 
	Mit Jsonix kann man ein XML Schema nehmen und daraus eine Mapping-Datei erzeugen.
	Dieser Vortrag gibt eine Überblick von Jsonix und zeigt es Live in WMS, WFS, CSW sowie OL3 WPS Client Demos vor.}

%\abstractZehn{Oliver Tonnhofer}{Mapnik oder MapServer}{...wer macht die schönsten Karten}{Mit Mapnik und MapServer stehen zwei OpenSource Kartenrenderer zur Verfügung, die in Geschwindigkeit, Funktionsumfang und Bildqualität kaum Wünsche übrig lassen. Aber welche Software nehme ich für mein Projekt?}
\abstractZehn{Oliver Tonnhofer}{Mapnik oder MapServer}{Wer macht die schönsten Kartenß}{%
	Mit Mapnik und MapServer stehen zwei OpenSource Kartenrenderer zur Verfügung, die in Geschwindigkeit, 
	Funktionsumfang und Bildqualität kaum Wünsche übrig lassen. Aber welche Software nehme ich für mein Projekt?
	
	Der Vortrag geht auf die kleinen und großen Unterschiede zwischen Mapnik und MapServer ein. Für welche Einsatzzwecke ist MapServer besser geeignet? Was kann Mapnik besonders gut? Wie können die Renderer in Anwendungen und Server integriert werden? Gibt es überhaupt nennenswerte Unterschiede?}

\abstractEins{Jochen Topf}{Taginfo und wie es die Welt sah}{}{%
	Das OpenStreetMap-Projekt organisiert seine Daten über ein offenes Taggingschema. 
	Jeder kann neue Objekte und Tags erfinden und in der OSM-Datenbank speichern. 
	Dieser Vortrag stellt den Web-Dienst "taginfo" vor, der Informationen zu Tags aus 
	verschiedenen Quellen darstellt und Mappern so hilft, die Übersicht zu behalten. 
	Der Vortrag geht sowohl auf die alltägliche Nutzung als auch auf weniger bekannte Seiten des Dienstes ein.}
%

\newtimeslot{12:15}
% 12:15
%
\abstractZwei{Christian Mayer}{Serverseitiges JavaScript und GIS}{Mit Node.js und npm die Welt beherrschen}{Der Vortrag gibt eine allgemeine Einführung in die Node.js-Welt und zeigt wie mit Node.js geo-relevante Probleme, wie das Einlesen von GIS-Datenformaten sowie die Verarbeitung von Geodaten möglich wird. Ebenso wird eine Übersicht von nützlichen Node.js-Modulen aus dem GIS-Bereich sowie Beispielanwendungen vorgestellt.}

%\abstractZehn{Daniel Koch}{GeoServer in action}{Fortgeschrittene Möglichkeiten beim Einsatz des GeoServers}{GeoServer in action - Fortgeschrittene Möglichkeiten beim Einsatz des GeoServers. Es geht um Kompilieren, Schnittstellenverwendung, Extensions, Performance-Tuning, GeoWebCache-Einsatz, Troubleshooting und Stolperfallen.}
\abstractZehn{Daniel Koch}{GeoServer in action}{Fortgeschrittene Möglichkeiten beim Einsatz des GeoServers}{%
	Der GeoServer ist ein weithin bekannter und mächtiger OpenSource Kartenserver. Sofern man die Umgebung eingerichtet hat, ist sowohl die Installation als auch die Konfiguration von ersten WMS- und WFS-Layern sehr einfach. In diesem Vortrag wird auf die typischen Anforderungen eines "GeoServers in action" eingegangen. Der Vortrag fokussiert sich also weniger auf die "Out-of-the-box"-Verwendung des GeoServers, sondern beleuchtet vielmehr fortgeschrittene Möglichkeiten beim Einsatz dieser Software.
	
	Themen des Vortrags sind das Kompilieren, die Verwendung der REST-Schnittstelle, Extensions, Performance-Tuning, GeoWebCache-Einsatz, Troubleshooting und Stolperfallen.}

%\abstractEins{Peter Barth}{Crowd-Sourced Elevation}{DIY DEM}{Dieser Vortrag zeigt einen neuen Ansatz auf, wie mit handelsüblichen Smartphones ein hochgenaues digitales Höhenmodell mit Hilfe von Crowd-Sourcing erzeugt werden kann, das sowohl hinsichtlich Auflösung als auch Genauigkeit SRTM weit überlegen ist.}
\abstractEins{Peter Barth}{Crowd-Sourced Elevation}{DIY DEM}{%	
	Digitale Höhenmodelle können für eine Vielzahl von Anwendungen im OpenStreetMap-Umfeld verwendet werden. 
	In Renderstilen werden sie z.\,B. für Schummerungen oder Höhenlinien verwendet. 
	Aber auch bei Routinganwendungen gibt es verschiedenste Gründe, das Geländemodell mit zu beachten: 
	Fahrradfahrer, Rollstuhlfahrer aber auch Energiesparen bei Elektromobilität.
	
	Mit SRTM steht ein genaues Höhenmodell mit nahezu globaler Abdeckung gemeinfrei zur Verfügung und 
	erst kürzlich wurde auch die Version mit einer Auflösung von einer Bogensekunde freigegeben. Aber reicht das?
	
	Nach einem Überblick über das Thema Höhenmodelle und die verschiedenen Datenquellen wird anhand von Experimenten und 
	deren Auswertung gezeigt, wie man mit einem handelsüblichen Smartphone selbst ein Höhenmodell erstellen 
	kann. Eignen sich die Methoden für Crowdsourcing und wie gut ist das Ergebnis verglichen mit SRTM?}
%

\newtimeslot{13:45}
% 13:45
%
\abstractAula{Arnulf Christl}{FOSSGIS-Zukunftswerkstatt}{Menschen begeistern im FOSSGIS-Verein mitzuwirken}{%
	Wer steckt hinter dem FOSSGIS e.V.? Um diese Frage zu beantworten veranstaltet der FOSSGIS e.V eine Zukunftswerkstatt. 
	Erfahren Sie mehr über Ziele, Strukturen und Menschen im Verein. 
	Lassen Sie sich begeistern von der Idee gemeinsam etwas zu bewegen. }
%
% 14:00
%
%

\newtimeslot{14:30}
% 14:30
%
\abstractZwei{Jakob Tworek}{Erfassung von Landnutzungsveränderungen mit FOSS Image Processing Tools}%
{%Change Detection mit QGIS - Landnutzungs- und -bedeckungsveränderungen erfassen
	}{%
	In diesem Vortrag wird eine Masterarbeit am Geographischen Institut der Universität Bonn vorgestellt. Im Rahmen dieser Masterarbeit 
	wurde zwei Change-Detection-Verfahren mit freier Bildverarbeitungssoftware angewandt und evaluiert.
	Zu den verwendeten Programmen zählen das Semi-Automatic Classification Plugin und Image Processing Tools der Orfeo Toolbox 
	und SAGA im Rahmen des Processing Framework von QGIS.
	}

\abstractZehn{Daniel Kastl}{Routing in der Datenbank}{Kürzeste-Wege-Berechnung und mehr mit pgRouting}{Die pgRouting-Erweiterung ermöglicht es, auf Daten in einer PostgreSQL-Datenbank eine kürzeste-Wege-Suche und andere netzorientierte Algorithmen anzuwenden. Neben den etablierten Funktionen sind für die nahe Zukunft eine Reihe neuer Algorithmen zur Tourenplanung zu erwarten. Dieser Vortrag stellt die verschiedenen Algorithmen vor und geht darauf ein, wie die Struktur der Netzdaten die Leistung des Systems beeinflussen kann.}

\abstractEins{Sascha L. Teichmann}{MTSatellite}{Echtzeit Webmapping für Minetest-Welten oder Spiel-Spass mit GIS}{%
	Der Vortrag stellt die Freie Software MTSatellite vor, ein Live-Webmapping System für das Open World/Sandbox-Spiel Minetest 
	(eine Open-Source-Alternative zu Minecraft). 
	Der Vortrag führt das System vor und gibt eine teils vertiefende Übersicht über die eingesetzen Technologien sowohl aus GIS- 
	als auch aus Sicht eines passionierten Minetest-Spielers.}
%

\newtimeslot{15:00}
% 15:00
%
\abstractZwei{Jens Schaefermeyer}{Automatisiertes Geodatenmanagement mit  GeoKettle}{}{%
	Dieser Vortrag stellt verschiedene Einsatzmöglichkeiten der freien ETL-Software GeoKettle vor. 
	GeoKettle ist die Open Source-Alternative zur verbreiteten Software "FME" und kann nicht nur Geodatenformate konvertieren, 
	sondern beispielsweise auch Objekte verteilen und zusammenfassen, redundante Daten 
	finden oder Prozesse in einer grafischen Oberfläche modellieren.}

\abstractZehn{Felix Kunde}{PostGIS Memento}{Versionierung von PostGIS-Datenbanken}{Memento. Gibt es nicht einen Film, der so heißt? Jemand, der kein Gedächtnis hat und sich alle Ereignisse aufschreibt? Zumindest geht es bei pgMemento darum. pgMemento zeichnet alle Veränderungen in einer PostgreSQL Datenbank auf und erlaubt die Wiederherstellung beliebiger frühere Zeitstände.}

%\abstractEins{Tobias Knerr}{Straßenrennen in der Innenstadt}{SuperTuxKart-Strecken bauen mit OSM2World}{%
%	Der Gründer des Open-Source-Projekts OSM2World erzählt von einer weiteren spannenden Anwendungsmöglichkeit des 3D-Renderers -- die 
%	Erstellung von Rennstrecken für das freie Computerspiel SuperTuxKart.	
%	}
\abstractEins{Tobias Knerr}{Straßenrennen in der Innenstadt}{SuperTuxKart-Strecken bauen mit OSM2World}{%
	Der Gründer des Open-Source-Projekts OSM2World erzählt von einer weiteren spannenden Anwendungsmöglichkeit des 3D-Renderers -- die 
	Erstellung von Rennstrecken für das freie Computerspiel SuperTuxKart.
	
	Der Vortrag beschreibt die Vorgehensweise, um die OSM-Daten einer Stadt in eine Rennstrecke zu verwandeln. Voraussetzung ist eine ausreichend gute Abdeckung der jeweiligen Region mit 3D-Attributen in OpenStreetMap.
	
	Aus diesen Daten wird mit dem 3D-Renderer OSM2World ein realitätsnahes 3D-Modell erstellt, das optional auch mit einem Geländemodell ergänzt werden kann. Mit dem 3D-Modelling-Tool Blender noch einige Verbesserungen vorgenommen und der gewünschte Streckenverlauf eingezeichnet. Ein Blender-Plugin erlaubt schließlich den Export einer fertigen Strecke für SuperTuxKart.	
}
%

\newtimeslot{16:00}
% 16:00
%
\abstractZwei{Till Adams}{WPS, GeoServer und SHOGun}{}{Der Vortrag stellt die Erweiterung des SHOgun Frameworks als WPS-CLient dar. Als WPS Server kommt der GeoServer-WPS zum Einsatz. Der Vortrag wird zum einen kurz die Mechanismen des WPS erläutern, die Einbindung in das SHOGun Framework an praktischen Beispielen zeigen und am Ende die Möglichkeiten, die sich daraus für ein WebGIS ergeben vorstellen. }

\abstractZehn{Bjoern Schilberg}{Docker für den GIS-Einsatz}{}{Mit der leichtgewichtige Virtualisierungs-Software Docker schnell und einfach Anwendungs-Container für Ihre GIS-Fachanwendung erstellen. Haben die GIS-Admins demnächst wieder mehr Zeit für ungleich spannendere Aufgaben als die Installation von Fachanwendungen?}

%\abstractEins{Vladimir Elistratov}{Prokitektura: prozedurale realistische 3D Gebäude und Städte}{}{Prokitektura is ein prozedurales und iteratives Verfahren für Schaffung architektonischer 3D Modelle. Ein Satz kleiner Funktionen auf Python wird verwendet um 3D Gebäude zu generieren.}
\abstractEins{Vladimir Elistratov}{Prokitektura -- prozedurale realistische 3D Gebäude und Städte}{}{%
	Prokitektura is ein prozedurales und iteratives Verfahren für Schaffung architektonischer 3D Modelle. 
	
	Ein Satz kleiner Funktionen auf Python wird verwendet um 3D Gebäude zu generieren.
	Eine solche Funktion nennt man Regel. Jede nachfolgende Regel verfeinert das Modell 
	und ergänzt es mit zusätzlichen Details. Zur Zeit ist Prokitektura als ein Add-on für 
	Open Source 3D Plattform Blender realisiert.}
%

\newtimeslot{16:30}
% 16:30
%
\abstractZwei{Astrid Emde}{Mapbender\,3 für den einfachen Aufbau von WebGIS Anwendungen}{So einfach ist der Aufbau von WebGIS Anwendungen mit Mapbender3}{Mapbender3 ist eine Software zur einfachen Erstellung von WebGIS Anwendungen. Der Vortrag geht vor allem auf die neuen Komponenten in Mapbender3 ein und stellt diese vor. Anpassung an das eigne Layout, Suchen in eignen Daten und Rechteverwaltung werden thematisiert.}

% Der kurze Abstract ist auch gut.
%\abstractZehn{Volker Mische}{GeoCouch}{Ein multidimensionaler Index für Couchbase}{%
%	GeoCouch ist ein multidimensionaler Index für Couchbase, einer verteilten dokumentenorientierten Datenbank. 
%	Dieser Vortrag wird eine kurze Einführung in Couchbase, die räumliche Abfragen ermöglicht, 
%	und die Datenmodellierung geben und dann in eine Live-Demonstration übergehen. }
\abstractZehn{Volker Mische}{GeoCouch}{Ein multidimensionaler Index für Couchbase}{%
	Couchbase ist eine verteilte dokumentenorientierten Datenbank und gehört damit 
	zur Kategorie der NoSQL Datenbanken. Das bedeutet, dass das Datenmodell nicht 
	relational ist, also ein gewisses Umdenken bei der Strukturierung der Daten nötig ist. 
	GeoCouch ist in Couchbase integriert und ermöglicht räumliche Abfragen. Diese sind nicht 
	auf zwei Dimensionen begrenzt, sondern können durch weitere Attribute erweitert werden. 
	So ist es möglich Faktoren, wie Zeit, Ausmaß oder Kategorien einzubeziehen. Eine beispielhafte 
	multidimensionale Datenbankabfrage wären alle Einwohner eines bestimmten Gebiets, die unter 30 
	Jahre alt sind und in einer Mietwohnung mit weniger als 50m$^2$ wohnen. Dieser Vortrag wird eine 
	kurze Einführung in Couchbase und die Datenmodellerung geben und dann in eine 
	Live-Demonstration übergehen.}

\abstractEins{Hongchao Fan}{Die neue WebGL-basierte Plattform für OSM-3D}{}{%
	In diesem Vortrag wird die technische Archtektur, das Funktiondesign und die 
	Implementierung der neue WebGL-basierte Plattform von OSM-3D präsentiert. 
	Es werden interaktive Visualisierungen der 3D Stadtmodelle und 3D Geländemodelle 
	demonstriert. Das neue Interface für interaktive Erfassungen von Dach- und 
	Fassadenstrukturen wird vorgestellt und mit Beispielen demonstriert. 
	Diskussion von Vorschlägen zur Verbesserungen der neuen Plattform. }
%

\newtimeslot{17:00}
% 17:00
%
%\abstractZwei{Simon Jirka}{Erfahrungen mit Sensor Web-Anwendungen}{%
%Erfahrungen mit Sensor Web-Anwendungen für mobile Geräte und Desktop-Anwendungen}{%
%Dieser Vortrag stellt einen freien, auf JavaScript basierenden Sensor Web-Client vor, 
%der auf allen Arten von Endgeräten von Mobiltelefonen bis hin zu Desktop-PCs einsetzbar 
%ist. Er geht auf die Entwicklung des Clients und auf seine Nutzung als eigenständige 
%Software und als Basis eigener Projekte ein.}
\abstractZwei{Simon Jirka}{Erfahrungen mit Sensor Web-An\-wendungen für mobile Geräte und Desktop-Anwendungen}{}{%
	Dieser Vortrag stellt einen freien, auf JavaScript basierenden Sensor Web-Client vor, 
	der auf allen Arten von Endgeräten von Mobiltelefonen bis hin zu Desktop-PCs einsetzbar ist. 
	Er geht auf die Entwicklung des Clients und auf seine Nutzung als eigenständige Software und 
	als Basis eigener Projekte ein.}

\abstractZehn{Johannes Weskamm}{Vector Tiles}{Performante Übertragung von umfangreichen Vektordaten}{%
	Es werden die Vor- und Nachteile von Vectortiling und der Vektordaten-Prozessierung im Client allgemein beleuchtet. 
	Als praktischer Anwendungsfall wird das Prinzip der Vectortiles anhand des Software-Stacks 
	TileStache, OpenLayers3 und PostGIS auf Basis von OSM-Daten vorgestellt.}

\abstractEins{Christoph Hormann (Moderation)}{OSM Lightning Talks}{}{}
%

\newsmalltimeslot{17:30}
% 17:30
\vspace{-2em}
\abstractZwei{\label{bof-donnerstag}Till Adams}{SHOGun BoF/Anwender- und Entwicklertreffen}{}{Es wird der aktuelle Entwicklungsstand von SHOgun, neue Ideen und auch neue Entwicklungen, die seit der FOSSGIS 2014 dazu gekommen sind, vorgestellt.}
\abstractZehn{Danilo Bretschneider}{deegree Anwendertreffen}{}{deegree ist ein Open-Source Java-Framework für die Verwaltung und Darstellung OGC-konformer Geoinformationssysteme.}
\abstractEins{Olaf Knopp}{PostNAS Anwendertreffen}{}{%
	PostNAS ist ein Projekt zum Import von ALKIS Daten über OGR. Es werden aktuelle Entwicklungen des Projekt vorgestellt.}
%

\newsmalltimeslot{19:30}
% 19:00
%
\abstractSenatssaal{Marco Lechner}{Vollversammlung FOSSGIS e.V.}{}{}