% 10:30
\renewcommand{\konferenztag}{\mittwoch}
\label{mittwoch}
\newtimeslot{10:30}
%
\abstractEins{Dominik Helle}{Was ist Open Source, wie funktioniert das?}{}{Die Organisation der Open Geo- und GIS-Welt. Worauf man achten sollte.}{Open Source hat viele Facetten - und es ranken sich inzwischen ebenso viele Mythen darum. Was davon richtig ist und was nicht stellen wir in einer kurzen Einführung zusammen. Der Vortrag richtet sich an alle, die mit Open Source bisher noch wenig Kontakt hatten und die Grundlagen verstehen möchten.}
%
% 13:00
\newtimeslot{13:00}
%
\abstractAula{Marco Lechner}{Eröffnung}{}{Eröffnung der FOSSGIS-Konferenz 2015 mit Begrüßung durch den Veranstalter (FOSSGIS e.V.), Grußworte der gastgebenden Universität. Bekanntgabe von organisatorischen Informationen für die Teilnehmer. Vortrag vom Goldsponsor.}
%
% 13:30
\newtimeslot{13:30}
%
\abstractAula{Frederik Ramm}{Nicht zuschauen - Mitmachen!}{}{10 Arten, wie Sie OpenStreetMap voranbringen können}{OpenStreetMap lebt vom Mitmachen. Dieser Vortrag zeigt einige naheliegende und einiger weniger naheliegende Wege auf, wie auch Sie OpenStreetMap weiter voranbringen können - als Privatperson, als Unternehmen, oder als Teil der öffentlichen Hand. }
%
% 14:00
\newtimeslot{14:00}
%
\abstractAula{Till Adams}{Softwarewartung für OpenSource. Ein Widerspruch?}{}{In meinem Vortrag möchte ich ein Thema diskutieren, das sowohl die Open Source Welt, als auch die kleine, heile FOSSGIS-Welt und damit natürlich auch uns bei terrestris seit einiger Zeit umtreibt: Softwarewartung für Open Source!}
%
% 14:30
\newtimeslot{14:30}
%
\abstractAula{Arndt Brenschede}{Lightning Talks}{}{Civic Tech in Deutschland, Intermodales ÖPNV/Rad/Fuss Routing, Hierarchische Geometrien für OSM Daten und FOSSGIS in der Hochschule}
%
% 15:30
\newtimeslot{15:30}
%
\abstractZwei{Jürgen Weichand}{Herausforderungen bei der Umsetzung der INSPIRE-Richtlinie}{}{Von Datenspezifikationen, komplexen Feature-Modellen, Open Source Softwareprodukten für INSPIRE und Darstellungs- und Dounloaddiensten.}
\abstractZehn{Horst Düster}{Neues von QGIS 2.8}{}{Im Vortrag werden die wichtigsten neuen Features von QGIS 2.6 und der kommenden Version 2.8 vorgestellt. Es wird ein Blick auf die Entwicklung in Richtung QGIS 3 geworfen.}
\abstractEins{Marc Tobias Metten}{Welches Münster meinen sie?}{Wie man Geocodern beibringt Orte mit gleichen Namen richtig zu sortieren}{Wie sortiert man Orte nach Wichtigkeit, wenn sie den gleichen Namen haben? Wir schauen uns Daten zu Suchverhalten, Größe, Bevölkerungsdichte, Tagging in OpenStreetMap und Verlinkung in Wikipedia/Wikidata anhand des Nominatim Geocoders an.}
%
% 16:00
\newtimeslot{16:00}
%
\abstractZwei{Armin Retterath}{GeoPortal.rlp}{Ein Erfolgsmodell für den Einsatz von FOSS in der öffentlichen Verwaltung}{Dieser Vortrag illustriert anhand der Entstehungsgeschichte des rheinland-pfälzischen Geoportals, das mittlerweile auch im Saarland und in Hessen eingesetzt wird, welches immense Potential im Einsatz von FOSS-Software in der öffentlichen Verwaltung steckt. Der Vortrag diskutiert neben unerwarteten positiven Nebenwirkungen wie der verstärkten Zusammenarbeit der Verwaltungen auch Risiken und Probleme, die in der 9-jährigen Geschichte des Projekts zu bewältigen waren.}
\abstractZehn{Pirmin Kalberer}{QGIS Plugins}{Must-Haves, Fachlösungen und Geheimtipps}{Eine grosse Stärke von QGIS ist die einfache, aber umfassende Erweiterbarkeit mittels Python Plugins. In kompakter Form wird eine Selektion aus der grossen Menge an öffentlich verfügbaren Plugins aus verschiedensten Bereichen vorgestellt.}
\abstractEins{Thomas Brinkhoff}{Offene Geodaten-Lage von Orten im Vergleich}{}{Offene Geodaten bezüglich der Lage von Orten (GeoNames Geographical Database, OSM, Wikipedia) werden vergleichend miteinander und in Bezug zu offenen amtlichen Geodaten untersucht.}
%
% 16:30
\newtimeslot{16:30}
%
\abstractZwei{David Arndt}{Geonetzwerk metropoleRuhr}{Die Geoinformationsplattform für das Ruhrgebiet}{Das Geonetzwerk metropoleRuhr ist eine Kooperation der Kreise und kreisfreien Städte im Ruhrgebiet mit dem Regionalverband Ruhr. Ziele sind GDI-Aufbau, Präsentation von Geodaten im eigenen Geoportal, Metadatenpflege, gegenseitige technische Unterstützung und gemeinsame Umsetzung der INSPIRE-Richtlinie.}

\abstractZehn{Andreas Neumann}{Neues vom QGIS Print Composer}{}{Der Vortrag fasst die in den letzten QGIS-Versionen eingefuehrten Verbesserungen im Print Composer 
	und im Atlas-Seriendruck zusammen. Pläne für die Entwicklung einer Reporting-Engine in einer zukünftigen QGIS-Version 
	werden vorgestellt.}


\abstractEins{Tim Alder}{OSM-Tagging in Wikidata}{}{Der Vortrag soll beschreiben wie das OSM-Taggingschemata in Wikidata kam, wozu das Ganze zukünftig gut sein könnte und welche Unterschiede es zwischen beiden Projekten gibt. }
%
% 17:00
\newtimeslot{17:00}
%
\abstractZwei{Tim Balschmiter}{Spatial Index von Solr }{Erfahrungen und Einblicke im Umgang mit Solr Spatial Index am Beispiel des Geoportal.de}{Seit der Version 4 bietet Solr Funktionen eines umfangreichen Spatial Indexes an. Am Beispiel des Geoportal.de sollen Erfahrungen und Einblicke in die Verwendung des Spatial Index mit Solr gegeben werden. }
\abstractZehn{Marco Bernasocchi}{ QField for QGIS}{Eine native Benutzerschnittstelle für mobile touch Geräte}{QField ist die native Benutzerschnittstelle für mobile touch Geräte und bietet eine vollwertige mobile GIS Daten Verwaltungsinfrastruktur. Das Synchronisationtool ermöglicht einen nahtlose Datenaustausch zwischen dem mobilen Gerät und der vorhandenen Infrastruktur.}
\abstractEins{Klaus Stein}{Ein GeoWiki auf OSM-Basis}{}{Wikis bieten editierbare Informationen in Text und Bild, OSM geographische Objekte u.a. in Kartenform. Unser GeoWiki leistet eine enge Integration beider Konzepte, indem Textstellen direkt und interaktiv auf Geoobjekte verweisen und umgekehrt.}
%
% 17:30
\newtimeslot{17:30}
%
\abstractZwei{Olaf Knopp}{Mapbender Anwendertreffen}{}{}
\abstractZehn{Danilo Bretschneider}{xPlanBox Anwendertreffen}{}{}
\abstractEins{Alexey Valikov}{ows.js Entwickler-Treffen}{Entwicklung der JS-Bibliothek für OWS-Clients.}{ows.js ist ein neues OSGeo-Projekt, mit dem Ziel, eine Client-Bibliothek für die OGC Web Services in JavaScript zu implementieren.  Auf dem Entwickler-Treffen werden wir über die Anforderungen, Erfahrungen, Architektur und Entwicklung sprechen.}