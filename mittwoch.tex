% 10:30
\renewcommand{\konferenztag}{\mittwoch}
\newtimeslot{10:30}
%
\abstractEins{Dominik Helle}{Was ist Open Source, wie funktioniert das?}{Die Organisation der Open Geo- und GIS-Welt. Worauf man achten sollte.}{Open Source hat viele Facetten -- und es ranken sich inzwischen ebenso viele Mythen darum. Was davon richtig ist und was nicht stellen wir in einer kurzen Einführung zusammen. Der Vortrag richtet sich an alle, die mit Open Source bisher noch wenig Kontakt hatten und die Grundlagen verstehen möchten.
	
	Open Source ist auf der einen Seite ein Entwicklungsmodell und auf der anderen ein Lizenzmodell. Zusammen bilden sie eine Kultur offener Entwicklungsgmemeinschaften, die höchst effektiv arbeiten. Diese Kultur ist um ein Vielfaches effektiver, als proprietäre Modelle es je sein können. Unter anderem sieht man das an einfachen Beispielen: Das Betriebssystem des Herstellers Apple basiert vollständig auf dem Open Source Unix FreeBSD. Es gibt halt einfach nichts besseres, und es selbst herzustellen wäre unendlich teuer. Sogar der hyper-proprietäre Hersteller Apple hat das eingesehen.
	
	Der Vortrag stellt die Geschichte der Entwicklung von Open Source vor und geht auf wichtige Grundlagen ein.
	}
%
% 13:00
\newtimeslot{13:00}
%
\abstractAula{Dominik Helle und Olaf Knopp}{Eröffnung}{}{Eröffnung der FOSSGIS-Konferenz 2015 mit Begrüßung durch den Veranstalter (FOSSGIS e.V.), Grußworte der gastgebenden Universität. Bekanntgabe von organisatorischen Informationen für die Teilnehmer. Vortrag des Goldsponsors WhereGroup GmbH.}
%
% 13:30
\newsmalltimeslot{13:30}
%
%TODO kürzen
\abstractAula{Frederik Ramm}{Nicht zuschauen -- Mitmachen!}{Zehn Arten, wie Sie OpenStreetMap voranbringen können}{OpenStreetMap lebt vom Mitmachen. Dieser Vortrag zeigt einige naheliegende und einiger weniger naheliegende Wege auf, wie auch Sie OpenStreetMap weiter voranbringen können -- als Privatperson, als Unternehmen, oder als Teil der öffentlichen Hand. 
	
	
	
	% Dass jeder bei OSM einen Account anmelden kann, ist bekannt - trotzdem schadet es nichts, das ab 
	% und zu zu wiederholen, und vor allem auf neue, einfachere Möglichkeiten der Datenerhebung hinzuweisen. 
	Viele mögen vielleicht denken, "es ist doch schon alles da", aber weit gefehlt -- jeder hat noch irgendwas im Kopf, was bei OSM nützlich sein kann.
	
	Das direkte Beisteuern von Daten ist aber nur einer von vielen Wegen, OSM nützlich zu sein. Auch "indirekte" Datenbereitstellungen -- Luftbilder, Vergleichsdaten, Straßenlisten usw. -- können dem Projekt nützen.
	
	Hacker lassen sich von den OSM-Daten zu neuer Software inspirieren und demonstrieren damit, was mit den freien Daten alles möglich ist. Sogar einige Unternehmen entwickeln mittlerweile Open Source Software, die in OSM eingesetzt werden kann, sei es zur Erfassung oder Nutzung der Daten - auch das bringt das OSM-Universum voran.
	
	Privatleute und Organisationen werden Mitglied im FOSSGIS e.V. oder der OSM Foundation und stärken so die Position dieser institutionellen Stützen des OSM-Projekts. Interessierte bringen sich in die Arbeit in verschiedenen Gremien der Vereine ein - zum Beispiel bei den "Working Groups" der OSM Foundation, die immer Leute suchen.
	
	Nicht zuletzt helfen natürlich auch Geldspenden an den FOSSGIS e.V. oder die OSM Foundation, wichtige Investitionen für das Projekt zu decken.
	
	%Dieser Vortrag stellt die verschiedenen Möglichkeiten vor, OSM zu helfen - in der Hoffnung, dass für jede(n) im Publikum etwas dabei ist.
	}
%
% 14:00
\newsmalltimeslot{14:00}
%
%TODO Kürzen
\abstractAula{Till Adams}{Softwarewartung für OpenSource -- ein Widerspruch?}{}{In meinem Vortrag möchte ich ein Thema diskutieren, das sowohl die Open Source Welt, als auch die kleine, heile FOSSGIS-Welt und damit natürlich auch uns bei terrestris seit einiger Zeit umtreibt: Softwarewartung für Open Source!
	
	
	
	Wenn es darum geht, Open Source Software einzusetzen, wird dem oft das Argument entgegengesetzt, das ja keiner verantwortlich sei, das es keine Betriebsgarantie gibt und die Entwickler ja "morgen schon was anderes machen könnten". Ich sehe dies als letzte Bastion der proprietären Hersteller im kürzlich als zu Ende erklärtem Glaubenskrieg zwischen proprietärer und Open Source Software-Verfechtern.
	
	Ich möchte in meinem Vortrag nicht diskutieren, inwieweit Architekturwechsel proprietärer Hersteller in der Vergangenheit dazu geführt haben, das Unsummen an investiertem Geld trotz sogenannter Betriebssicherheit unwiderbringlich den Rhein herabgeflossen sind (Stichwort ArcView GIS, Windows XP u.\,v.\,m.). Trotzdem wird solchen Anbietern eher zugetraut, das eine angebotene Softwarewartung zu Betriebs- und damit Investitionssicherheit beiträgt. Im Vortrag steht vielmehr die Frage im Raum, ob es eine Verletzung des Open Source Grundsatzes ist, wenn eine Firma, die maßgeblich hinter der Entwicklung einer oder mehrerer Open Source (GIS-)Projekten steht, eine solche Wartung nach proprietärem Geschäftsmodell zu einem jährlichen Fixpreis anbietet? Kann das Angebot von Betriebssicherheit und auch Support dazu führen, das Open Source eher eingesetzt wird? Würde ein solches Angebot überhaupt Chancen beim Kunden haben -- da sie ja anders als beim proprietären Geschäftsmodell nicht obligatorisch wäre (sein kann!)? De Facto bieten Firmen solche Modelle bereits an, es stellt sich die Frage, ob der Markt reif ist für diese nächste "Professionalisierungsstufe"?
	
	All dies sind Fragen die ich in meinem Vortrag behandeln und anschließend gerne auch diskutieren möchte.
	}
%
% 14:30
\newsmalltimeslot{14:30}
%
% \abstractAula{Arndt Brenschede}{Lightning Talks}{}{Civic Tech in Deutschland, Intermodales ÖPNV/Rad/Fuss Routing, Hierarchische Geometrien für OSM Daten und FOSSGIS in der Hochschule}

% Für Aufzählung Abstand zwischen Aufzählungspunkten als Default in separater Variable speichern
\newlength{\itemsepdefault}
\setlength{\itemsepdefault}{\itemsep}
\abstractAula{}{Lightning Talks}{}{Es sprechen:
	
	\begin{itemize}
		\setlength{\itemsep}{-2pt} % Aufzählungspunktabstand auf 0
		\item Arndt Brenschede über Intermodales ÖPNV-Rad-Fuß-Routing
		\item Arnulf Christl über %TODO
		\item Mila Frerichs über %TODO
		\item Robert Buchholz über %TODO
	\end{itemize}
	}
%
% 15:30
\newtimeslot{15:30}
%
\abstractZwei{Jürgen Weichand}{Herausforderungen bei der Umsetzung der INSPIRE-Richtlinie}{}{Der Vortrag handelt von Datenspezifikationen, komplexen Feature-Modellen, Open Source Softwareprodukten für INSPIRE und Darstellungs- und Dounloaddiensten.
	
	Wie unterscheiden sich einfache und komplexe Feature-Modelle? Welche Lösungsansätze gibt es bei der Bereitstellung von komplexen Feature-Modellen? Mit welchen OpenSource-Softwareprodukten ist die Bereitstellung von INSPIRE-konformen Daten möglich?
	
	Weiterhin wird am Beispiel QGIS auf vorhandene Einschränkungen bei der Verwendung von Geodaten auf Grundlage von GML-Anwendungsschemata eingegangen. }
\abstractZehn{Horst Düster}{Neues von QGIS 2.8}{}{%
	Im Vortrag werden die wichtigsten neuen Features von QGIS 2.6 und der kommenden Version 2.8 vorgestellt. Es wird ein Blick auf die Entwicklung in Richtung QGIS 3 geworfen.
	
	Seit der FOSSGIS2014 in Berlin hat sich QGIS, dank der Investitionen der QGIS Anwender und der Spenden vieler Sponsoren, erheblich weiter entwickelt. Im Vortrag werden die wichtigsten neuen Features von QGIS 2.6 und der kommenden Version 2.8 vorgestellt. Ausserdem wird ein Blick auf die künftige Entwicklung in Richtung QGIS 3 geworfen.}

\abstractEins{Marc Tobias Metten}{Welches Münster meinen sie?}{Wie man Geocodern beibringt Orte mit gleichen Namen richtig zu sortieren}{%
	Wie sortiert man Orte nach Wichtigkeit, wenn sie den gleichen Namen haben? Wir schauen uns Daten zu Suchverhalten, Größe, Bevölkerungsdichte, Tagging in OpenStreetMap und Verlinkung in Wikipedia/Wikidata anhand des Nominatim Geocoders an.
	
	Der Nominatim Geocoder nutzt OpenStreetMap, minütlich aktualisiert. Die Suchergebnisse werden nach Relevanz sortiert. Einige Faktoren, wie dass Ländernamen wichtiger sind als Strassennamen sind jedem klar. Auf Mailinglisten kommt aber hin und wieder die Frage auf warum genau ein Ort wichtiger eingeschätzt wird als ein anderer.
	
%	Frankfurt gibt es zweimal, beide sind grosse Städte. Münster gibt es mehrfach. Sogar Paris, Berlin und Frankreich sobald man die ganze Welt betrachtet.
	
	Nominatim nutzt u.a. einen vorberechneten Score (numerischer Wert), der auf den Verlinkungen innerhalb Wikipedia basiert. Die erste Version sogar auf Seitenabrufen. Das hat Vor- und Nachteile (und Bugs). 
	%Leider sind selbst für Nutzer (Administratoren) von Nominatim die Algorithmen dahinter nicht transparent genug. Viele laden einfach eine selten aktualisierte Binärdatei von http://www.nominatim.org/.
	
%	Ich habe mich damit intensiver beschäftigt und plane bis zur FOSSGIS die Scripte neu schreiben und neu dokumentieren. Selbst wenn nicht wird der Vortrag ein guter Überlick werden. Ich arbeite bei http://data.opencagedata.com/index.html#about-section und wir bieten einen Geocoder u.a. auf Basis von Nominatim an http://geocoder.opencagedata.com/. Ich arbeite seit 2006 Jahren mit geocodern mit underschiedlichen kommerziellen und offenen Daten.
}

%
% 16:00
\newtimeslot{16:00}
%
%TODO könnte man noch ein klein wenig verlängern
\abstractZwei{Armin Retterath}{GeoPortal.rlp}{Ein Erfolgsmodell für den Einsatz von FOSS in der öffentlichen Verwaltung}{Dieser Vortrag illustriert anhand der Entstehungsgeschichte des rheinland-pfälzischen Geoportals, das mittlerweile auch im Saarland und in Hessen eingesetzt wird, welches immense Potential im Einsatz von FOSS-Software in der öffentlichen Verwaltung steckt. Der Vortrag diskutiert neben unerwarteten positiven Nebenwirkungen wie der verstärkten Zusammenarbeit der Verwaltungen auch Risiken und Probleme, die in der 9-jährigen Geschichte des Projekts zu bewältigen waren.}

%\abstractZehn{Pirmin Kalberer}{QGIS Plugins}{Must-Haves, Fachlösungen und Geheimtipps}{Eine grosse Stärke von QGIS ist die einfache, aber umfassende Erweiterbarkeit mittels Python Plugins. In kompakter Form wird eine Selektion aus der grossen Menge an öffentlich verfügbaren Plugins aus verschiedensten Bereichen vorgestellt.}

\abstractZehn{Pirmin Kalberer}{QGIS Plugins}{Must-Haves, Fachlösungen und Geheimtipps}{Eine grosse Stärke von QGIS ist die einfache, aber umfassende Erweiterbarkeit mittels Python Plugins. In kompakter Form wird eine Selektion aus der grossen Menge an öffentlich verfügbaren Plugins aus verschiedensten Bereichen vorgestellt. Die Auswahl geht vom bestens bekannten Open Layers Plugin und weiteren "Must-Haves" über wenig bekannte Core-Plugins wie dem Offline Editing Plugin bis zu Insidertipps wie dem Remote Debugging Plugin. Neben nützlichen Helfern werden auch umfangreiche Fachlösungen kurz vorgestellt. Es werden Anwendungsbereiche von der Datenerfassung bis zur Plugin-Entwicklung abgedeckt, aber auch Tipps gegeben, wie man selbst ein passendes Plugin findet und evaluiert.}

\abstractEins{Thomas Brinkhoff}{Offene Geodaten-Lage von Orten im Vergleich}{}{Offene Geodaten bezüglich der Lage von Orten (GeoNames Geographical Database, OSM, Wikipedia) werden vergleichend miteinander und in Bezug zu offenen amtlichen Geodaten untersucht.
	
	Untersucht wurden die Nutzbarkeit der Daten, d.\,h. Einfachheit des Zugriffs, bereitgestellte Attribute 
	und Zuordnungsmöglichkeiten zu einem gegebenen Ort,	die Vollständigkeit der Daten und die Qualität der Koordinaten.
	}
%
% 16:30
\newtimeslot{16:30}
%
\abstractZwei{David Arndt}{Geonetzwerk metropoleRuhr}{Die Geoinformationsplattform für das Ruhrgebiet}{Das Geonetzwerk metropoleRuhr ist eine Kooperation der Kreise und kreisfreien Städte im Ruhrgebiet mit dem Regionalverband Ruhr. Ziele sind GDI-Aufbau, Präsentation von Geodaten im eigenen Geoportal, Metadatenpflege, gegenseitige technische Unterstützung und gemeinsame Umsetzung der INSPIRE-Richtlinie.}

\abstractZehn{Andreas Neumann}{Neues vom QGIS Print Composer}{}{Der Vortrag fasst die in den letzten QGIS-Versionen eingefuehrten Verbesserungen im Print Composer 
	und im Atlas-Seriendruck zusammen. Pläne für die Entwicklung einer Reporting-Engine in einer zukünftigen QGIS-Version 
	werden vorgestellt.}


%\abstractEins{Tim Alder}{OSM-Tagging in Wikidata}{}{Der Vortrag soll beschreiben wie das OSM-Taggingschemata in Wikidata kam, wozu das Ganze zukünftig gut sein könnte und welche Unterschiede es zwischen beiden Projekten gibt. }
\abstractEins{Tim Alder}{OSM-Tagging in Wikidata}{}{%
	Wikidata ist die freie Datensammlung aus dem Wikipedia-Umfeld.
	
	Das Vorhaben das OSM-Tagging in Wikidata zu hinterlegen stellt eine Ergänzung zu dem Verlinken von OSM-Objekten mit Wikipedia-/Wikidata-Tags dar, es wird vielmehr versucht die übergeordneten OSM-Objektklassen mit Wikidata zu verbinden. Dazu wurde in Wikidata ein neues Property für das OSM-Tagging angelegt und in diesem wird das Tagging als Key oder Tag hinterlegt. 
	
	Der Vortrag wird ein paar Beispiele (Passstraße, Trafo, Tanzeinrichtung, Schuhverkäufer, Fahrradladen) aufführen, wo das Matching zwischen beiden Projekten schlecht funktioniert. Teilweise fehlen Wkipedia-Artikel. Eine 1:1-Umsetzung des OSM-Wikis kann der Links zur Wikipedia nicht sein, somit bleibt weiterhin das OSM-Wiki die verbindliche Beschreibungsseite des OSM-Taggings.
	 }
%
% 17:00
\newtimeslot{17:00}
%
%\abstractZwei{Tim Balschmiter}{Spatial Index von Solr }{Erfahrungen und Einblicke im Umgang mit Solr Spatial Index am Beispiel des Geoportal.de}{Seit der Version 4 bietet Solr Funktionen eines umfangreichen Spatial Indexes an. Am Beispiel des Geoportal.de sollen Erfahrungen und Einblicke in die Verwendung des Spatial Index mit Solr gegeben werden. }
\abstractZwei{Tim Balschmiter}{Spatial Index von Solr }{}{Seit der Version 4 bietet Solr Funktionen eines umfangreichen Spatial Indexes an. Am Beispiel des Geoportal.de sollen Erfahrungen und Einblicke in die Verwendung des Spatial Index mit Solr gegeben werden.
	
	Die GDI-DE stellt Beschreibungen und Zugangsdaten zu Geodaten von Bund, Ländern und Kommunen bereit, welche über das Geoportal.de recherchiert werden können. Über eine Volltextsuche können derzeit ca. 130000 Datensätze gefunden werden. Für einen performanten Zugriff auf die Datensätze wird die Open Source Software Solr eingesetzt. Derzeit erfolgt die Suche ausschließlich über die themenbezogenen Inhalte der Metadaten.
	}

% Was für ein Laberfassel-Abstract. Jedes Wort mehr ist heiße Marketingluft.
\abstractZehn{Marco Bernasocchi}{QField for QGIS}{}{%
QField ist die native Benutzerschnittstelle für tragbare Touchscreengeräte und bietet eine vollwertige mobile Geodaten-Verwaltungsinfrastruktur. 
Das Synchronisationtool ermöglicht einen nahtlosen Datenaustausch zwischen dem mobilen Gerät und der vorhandenen Infrastruktur.}

%\abstractEins{Klaus Stein}{Ein GeoWiki auf OSM-Basis}{}{Wikis bieten editierbare Informationen in Text und Bild, OSM geographische Objekte u.a. in Kartenform. Unser GeoWiki leistet eine enge Integration beider Konzepte, indem Textstellen direkt und interaktiv auf Geoobjekte verweisen und umgekehrt.}
\abstractEins{Klaus Stein}{Ein GeoWiki auf OSM-Basis}{}{%
	Wikis bieten editierbare Informationen in Text und Bild, OSM geographische Objekte u.a. in Kartenform. Unser GeoWiki leistet eine enge Integration beider Konzepte, indem Textstellen direkt und interaktiv auf Geoobjekte verweisen und umgekehrt.
	
	Die referenzierten und dargestellten Geoobjekte müssen dabei nicht zwingend in der OSM-Datenbank vorliegen, sondern können auch vom Nutzer im Wiki direkt angelegt werden. Zum einen senkt dies die Hemmschwelle etwas falsch zu machen, zum anderen erlaubt es die Nutzung des GeoWikis für private Daten.}
%
% 17:30
\newtimeslot{17:30}
%
\abstractZwei{Olaf Knopp}{Mapbender Anwendertreffen}{}{}
\abstractZehn{Danilo Bretschneider}{xPlanBox Anwendertreffen}{}{}
\abstractEins{Alexey Valikov}{ows.js Entwickler-Treffen}{Entwicklung der JS-Bibliothek für OWS-Clients.}{ows.js ist ein neues OSGeo-Projekt, mit dem Ziel, eine Client-Bibliothek für die OGC Web Services in JavaScript zu implementieren.  Auf dem Entwickler-Treffen werden wir über die Anforderungen, Erfahrungen, Architektur und Entwicklung sprechen.}