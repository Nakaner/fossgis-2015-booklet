% neuer Zeitslot
\newcommand{\talktime}{9:99}
\newcommand{\newtimeslot}[1]{\newpage\renewcommand{\talktime}{#1}}

% neuer Zeitslot ohne Seitenumbruch
\newcommand{\newsmalltimeslot}[1]{\renewcommand{\talktime}{#1}}

% \konferenztag initialisieren
\newcommand{\konferenztag}{KeinTag}


% Hintergrund setzen
\def\mittwoch{Mittwoch}
\def\donnerstag{Donnerstag}
\def\freitag{Freitag}
\newcommand{\setpagebackground}{ %
%	\ifthispageodd{\ThisCenterWallPaper{1.0}{freitag-ungerade}}{\ThisCenterWallPaper{1.0}{freitag-gerade}}%
	\ifthenelse{\equal{\konferenztag}{\mittwoch}}{ %
		\ifthispageodd{\ThisCenterWallPaper{1.0}{mittwoch-ungerade}}{\ThisCenterWallPaper{1.0}{mittwoch-gerade}}%
	}{}
	\ifthenelse{\equal{\konferenztag}{\donnerstag}}{ %
		\ifthispageodd{\ThisCenterWallPaper{1.0}{donnerstag-ungerade}}{\ThisCenterWallPaper{1.0}{donnerstag-gerade}}%
	}{}
	\ifthenelse{\equal{\konferenztag}{\freitag}}{ %
		\ifthispageodd{\ThisCenterWallPaper{1.0}{freitag-ungerade}}{\ThisCenterWallPaper{1.0}{freitag-gerade}}%
	}{}
}


\newcolumntype{Y}[1]{>{\raggedright\arraybackslash}p{#1}}

% Titeltabelle setzen
\newcommand{\setabstract}[6]{
	% 1. Sprecher
	% 2. Titel
	% 3. Untertitel
	% 4. Abstract (Text)
	% 5. Farbe
	% 6. Raum
	\thispagestyle{scrheadings} 
	\setlength\tabcolsep{0pt}
	% \setlength{\fboxsep}{0pt}
	\noindent\colorbox{#5}{%
		\noindent\begin{tabu}{X[5L]r}
			\emph{#1} % Sprecher
			&
			\talktime
			\tabularnewline
			{\par\noindent\large \sectfont #2}% % Titel
			&
			#6
			\tabularnewline
			\issubtitleempty{#3}
%			\multicolumn{2}{p{\linewidth}}{\issubtitleempty{#3}}
%			\tabularnewline
		\end{tabu}%
	}
	%
	\isabstractempty{#4}%
	\setpagebackground%
	\vspace{0.5em}% Abstand zum nächsten Talk, auch wenn es keinen Abstract gibt
}



% Setzen des Untertitels
% muss ausgelagert und durch \makeatletter umgeben sein
\makeatletter
\newcommand{\issubtitleempty}[1]{%
	\@ifnotmtarg{#1}{\multicolumn{2}{Y{\linewidth}}{\vspace{-0.6em} \noindent\bfseries \normalsize \sectfont #1}\tabularnewline}
}
\makeatother

% Setzen des Abstracts, falls vorhanden
% muss ausgelagert und durch \makeatletter umgeben sein
\makeatletter
\newcommand{\isabstractempty}[1]{%
		\vspace{0.5em}%
		\par\noindent #1 \par% % Abstract
		\vspace{1.5em}% Abstand zum nächsten Talk, auch wenn es einen Abstract gibt
}
\makeatother

% Abstrakt Saal S10
\newcommand{\abstractZehn}[4]%
{%
	\definecolor{zehn}{cmyk}{ 0 0.13 0.35 0.03}
	\setabstract{#1}{#2}{#3}{#4}{zehn}{S\,10}
}

% Abstract Saal S2
\newcommand{\abstractZwei}[4]%
{%
	\definecolor{zwei}{cmyk}{ .1 0 .17 .05}
	\setabstract{#1}{#2}{#3}{#4}{zwei}{S\,2}
}

% Abstract Saal S1
\newcommand{\abstractEins}[4]%
{%
	\definecolor{eins}{cmyk}{ 0 .18 .06 .10}
	\setabstract{#1}{#2}{#3}{#4}{eins}{S\,1}
}

% Abstract Aula
\newcommand{\abstractAula}[4]%
{%
	\setabstract{#1}{#2}{#3}{#4}{orange}{Aula}
}

% Abstract Senatssaal
%TODO Farbe ändern
\newcommand{\abstractSenatssaal}[4]%
{%
	\setabstract{#1}{#2}{#3}{#4}{orange}{Senatssaal}
}

% viel zu lang
\newcommand{\zulang}{Dieser Text ist viel zu lang. Dieser Text ist viel zu lang. Dieser Text ist viel zu lang. Dieser Text ist viel zu lang. Dieser Text ist viel zu lang. Dieser Text ist viel zu lang. Dieser Text ist viel zu lang. Dieser Text ist viel zu lang. Dieser Text ist viel zu lang. Dieser Text ist viel zu lang. Dieser Text ist viel zu lang. Dieser Text ist viel zu lang. Dieser Text ist viel zu lang. Dieser Text ist viel zu lang. }