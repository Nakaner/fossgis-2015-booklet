% neuer Zeitslot
\newcommand{\talktime}{9:99}
\newcommand{\newtimeslot}[1]{\newpage\renewcommand{\talktime}{#1}}

% neuer Zeitslot ohne Seitenumbruch
\newcommand{\newsmalltimeslot}[1]{\renewcommand{\talktime}{#1}}

% \konferenztag initialisieren
\newcommand{\konferenztag}{KeinTag}


% Hintergrund setzen
\def\mittwoch{Mittwoch}
\def\donnerstag{Donnerstag}
\def\freitag{Freitag}
\newcommand{\setpagebackground}{ %
%	\ifthispageodd{\ThisCenterWallPaper{1.0}{freitag-ungerade}}{\ThisCenterWallPaper{1.0}{freitag-gerade}}%
	\ifthenelse{\equal{\konferenztag}{\mittwoch}}{ %
		\ifthispageodd{\ThisCenterWallPaper{1.0}{mittwoch-ungerade}}{\ThisCenterWallPaper{1.0}{mittwoch-gerade}}%
	}{}
	\ifthenelse{\equal{\konferenztag}{\donnerstag}}{ %
		\ifthispageodd{\ThisCenterWallPaper{1.0}{donnerstag-ungerade}}{\ThisCenterWallPaper{1.0}{donnerstag-gerade}}%
	}{}
	\ifthenelse{\equal{\konferenztag}{\freitag}}{ %
		\ifthispageodd{\ThisCenterWallPaper{1.0}{freitag-ungerade}}{\ThisCenterWallPaper{1.0}{freitag-gerade}}%
	}{}
}


% Titeltabelle setzen
\newcolumntype{Y}[1]{>{\raggedright\arraybackslash}p{#1}}

%% Längen für Sponsorenbox und Titelboxen
\newlength{\titleboxwidth}
\setlength{\titleboxwidth}{\textwidth}
\advance\titleboxwidth by -6pt
\newcommand{\setabstract}[6]{
	% 1. Sprecher
	% 2. Titel
	% 3. Untertitel
	% 4. Abstract (Text)
	% 5. Farbe
	% 6. Raum
	\thispagestyle{scrheadings} 
	\setlength\tabcolsep{0pt}
	% \setlength{\fboxsep}{0pt}
	\noindent\fcolorbox{white}{#5}{\parbox{\titleboxwidth}{%
		\noindent\begin{tabu}{X[5L]r}
			\isspeakerempty{#1}{#2}{#6}
%			\emph{#1} % Sprecher
%			&
%			\talktime
%			\tabularnewline
%			{\par\noindent\large \sectfont #2}% % Titel
%			&
%			#6
%			\tabularnewline
			\issubtitleempty{#3}
%			\multicolumn{2}{p{\linewidth}}{\issubtitleempty{#3}}
%			\tabularnewline
		\end{tabu}%
	}}
	%
	\isabstractempty{#4}%
	\setpagebackground%
	\vspace{0.5em}% Abstand zum nächsten Talk, auch wenn es keinen Abstract gibt
	\setlength\tabcolsep{6pt} % Spaltenpadding wieder auf Default setzen
}

% Setzen des Referenten, falls vorhanden
% Wir gehen davon aus, dass es einen Untertitel nur dann gibt, wenn es auch einen Referenten gibt
\makeatletter
\newcommand{\isspeakerempty}[3]{%
	% Arguments:
	% 1. speaker
	% 2. title
	% 3. room
	\@ifmtarg{#1}{%
			\par\noindent\large \sectfont #2% % Titel
			&
			#3, \talktime
			\tabularnewline
		}
		{
			\emph{#1} % Sprecher
			&
			\talktime
			\tabularnewline
			{\par\noindent\large \sectfont #2}% % Titel
			&
			#3
			\tabularnewline
		}
		
}
\makeatother

% Setzen des Untertitels
% muss ausgelagert und durch \makeatletter umgeben sein
\makeatletter
\newcommand{\issubtitleempty}[1]{%
	\@ifnotmtarg{#1}{\multicolumn{2}{Y{\linewidth}}{\vspace{-0.6em} \noindent\bfseries \normalsize \sectfont #1}\tabularnewline}
}
\makeatother

% Setzen des Abstracts, falls vorhanden
% muss ausgelagert und durch \makeatletter umgeben sein
\makeatletter
\newcommand{\isabstractempty}[1]{%
		\vspace{0.5em}\newline%
		%\par\noindent 
		#1 \par% % Abstract
		\vspace{1.5em}% Abstand zum nächsten Talk, auch wenn es einen Abstract gibt
}
\makeatother

% Farben definieren
%\definecolor{eins}{cmyk}{ 0 .18 .06 .10}
\definecolor{eins}{cmyk}{ 0 .13 .04 .08}
\definecolor{zwei}{cmyk}{ .1 0 .17 .05}
\definecolor{hellorange}{cmyk}{ 0 0.13 0.35 0.03}
%\definecolor{aula}{cmyk}{ 0.2 0 0.05 0.13}
\definecolor{aula}{cmyk}{ 0.13 0 0.04 0.11}
\definecolor{geoblau}{cmyk}{ 0.2 .02 0 .01}
\definecolor{dezentrot}{cmyk}{ 0 .17 0.21 .04}
\definecolor{hellgelb}{cmyk}{ 0 .02 0.26 0}
\definecolor{hellgruen}{cmyk}{ 0.05 .0 0.17 0.05}

% Abstract Saal S10
\newcommand{\abstractZehn}[4]%
{%
	\setabstract{#1}{#2}{#3}{#4}{geoblau}{S\,10}
}

% Abstract Saal S2
\newcommand{\abstractZwei}[4]%
{%
	\setabstract{#1}{#2}{#3}{#4}{hellgelb}{S\,2}
}

% Abstract Saal S1
\newcommand{\abstractEins}[4]%
{%
	\setabstract{#1}{#2}{#3}{#4}{hellgruen}{S\,1}
}

% Abstract Aula
\newcommand{\abstractAula}[4]%
{%
	\setabstract{#1}{#2}{#3}{#4}{hellorange}{Aula}
}

% Abstract Senatssaal
%TODO Farbe ändern
\newcommand{\abstractSenatssaal}[4]%
{%
	\setabstract{#1}{#2}{#3}{#4}{dezentrot}{Senatssaal}
}

% Workshop STUDLAB
\newcommand{\workshopbox}[3]%
{%
	% 1. Titel
	% 2. Referent
	% 3. STUDLAB-Nummer3
	\setlength\tabcolsep{0pt}
	\noindent\fcolorbox{white}{dezentrot}{\parbox{\titleboxwidth}{%
			\noindent
			\begin{tabu}{X[5L]r}
				\emph{#2} % Sprecher
				&
				\talktime
				\tabularnewline
				{\noindent\large \bfseries #1}% % Titel
				&
				STUDLAB\,#3
				\tabularnewline
			\end{tabu}
		}
	}
	\setlength\tabcolsep{6pt} % Spaltenpadding wieder auf Default setzen
}

% viel zu lang
\newcommand{\zulang}{Dieser Text ist viel zu lang. Dieser Text ist viel zu lang. Dieser Text ist viel zu lang. Dieser Text ist viel zu lang. Dieser Text ist viel zu lang. Dieser Text ist viel zu lang. Dieser Text ist viel zu lang. Dieser Text ist viel zu lang. Dieser Text ist viel zu lang. Dieser Text ist viel zu lang. Dieser Text ist viel zu lang. Dieser Text ist viel zu lang. Dieser Text ist viel zu lang. Dieser Text ist viel zu lang. }

%% Längen für Sponsorenbox
\newlength{\fboxwidth}
\setlength{\fboxwidth}{\textwidth}
\advance\fboxwidth by -9pt
%% Sponsorenbox
%% 1. Logo
%% 2. Logobreite
%% 3. negativer vspace nach Logo
%% 4. Text
\newcommand{\sponsorenbox}[4]{%
	\setlength{\fboxsep}{4.5pt}
	\noindent\fcolorbox{gray}{white}{\parbox{\fboxwidth}{
			\begin{floatingfigure}[r]{#2}
				\centering
				\includegraphics[width=#2]{#1}
				\vspace{#3}
			\end{floatingfigure}		
			
			\noindent #4
		}}
		\setlength{\fboxsep}{3pt}
	}